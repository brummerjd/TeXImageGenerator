\documentclass[preview]{standalone}
\usepackage{amsmath}
\usepackage{enumerate}
\usepackage{tikz}

\begin{document}

% These are the parameters that will be randomized
%  They can be named anything, and there can be any number of them
\newcommand\inputzero{0} % 0, 1, 2, 3
\newcommand\outputzero{6} % 6, 7, 8
\newcommand\function{0} % 0, 1

% These parameters are calculated based on the randomized values
\pgfmathtruncatemacro\inputone{\inputzero+1}
\pgfmathtruncatemacro\inputtwo{\inputzero+1+\function}
\pgfmathtruncatemacro\inputthree{\inputzero+2+\function}
\pgfmathtruncatemacro\outputone{\outputzero+3}
\pgfmathtruncatemacro\outputtwo{\outputzero+3+(1-\function)}
\pgfmathtruncatemacro\outputthree{\outputzero+5+(1-\function)}

% Statement of the problem begins here
%  This is standard LaTeX code using the variables defined above
Consider the following table of values.

\begin{center}
\begin{tabular}{|c|c|c|c|c|}
	\hline
	$x$ & \inputzero & \inputone & \inputtwo & \inputthree \\
	\hline
	$f(x)$ & \outputzero & \outputone & \outputtwo & \outputthree \\
	\hline
\end{tabular}
\end{center}

\begin{enumerate}[a)]
	\setlength\itemsep{1em}
	\item Can you conclude that $f(\inputzero)=\outputzero$? Clearly explain your reasoning in a few sentences.
	\item Can you conclude that $f(\outputthree)=\inputthree$? Clearly explain your reasoning in a few sentences.
	\item Does the table above represent a function? Clearly explain your reasoning in a few sentences.
\end{enumerate}

\end{document}
